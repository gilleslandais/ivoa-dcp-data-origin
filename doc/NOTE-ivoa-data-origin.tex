\documentclass[11pt,a4paper]{ivoa}
\input tthdefs

\title{Data Origin in the VO}

% see ivoatexDoc for what group names to use here
\ivoagroup{DCP}

%\author[????URL????]{G.Landais}
\author{G.Landais}
\author{G.Muench}
\author{looking for contributors}
%\author{????Fred Offline????}

\editor{G.Landais}

% \previousversion[????URL????]{????Concise Document Label????}
\previousversion{This is the first public release}
       

\begin{document}
\begin{abstract}
The goal of the document is to make the Data Origin more visible in the query results executed in the Virtual Observatory. 
The document lists meta-data required to provide sufficient traceability to end-users in order to improve the understanding 
of the resultsets and enabling its reuse and its citation.


\textbf{NOTE} in work -  template for a possible IVOA note.
\end{abstract}


\section*{Acknowledgments}


\section*{Conformance-related definitions}


\section{Introduction}

Data origin is required for end users to understand data, for citation and for reusability. The  provenance is cited as a mandatory criterion in the EOSC or in RDA FAIR definition. 

The virtual observatory provides an advanced framework to search and consume data provided by Data Centers or Space Agencies who apply curation in different level.  In this context, Data Origin in output includes meta-data from the data producer (author, space agency) and the Data Center which hosts the resource. 
Depending of the implementation, the users can find the origin information in the Data center web pages (landing pages) or in the Registries of the Virtual Observatory. For citation, ADS (Astrophysics Data System, Nasa) provides citation capabilities with bibtex output. There are no VO standards to get the information easily yet. For instance, the origin meta-data are not included neither in output format, nor in protocols used to access the data.
A list of basics meta-data added in strategic location (as result output or resource listing) would give easier the authors search who is looking for how to cite VO resources. Tracing data origin, from the producer to the final query enables also to report to end users the different agents implied in the data preservation (authors, data center, space agencies, journal)- especially when data can be subject to a curation  which depends of the different agents.
We propose to list the meta-data which responds to the need of Provenance and methods available today for their implementations. 

\subsection{Role within the VO Architecture}

%\begin{figure}
%\centering

% As of ivoatex 1.2, the architecture diagram is generated by ivoatex in
% SVG; copy ivoatex/archdiag-full.xml to role_diagram.xml and throw out
% all lines not relevant to your standard.
% Notes don't generally need this.  If you don't copy role_diagram.xml,
% you must remove role_diagram.pdf from SOURCES in the Makefile.

%\includegraphics[width=0.9\textwidth]{role_diagram.pdf}
%\caption{Architecture diagram for this document}
%\label{fig:archdiag}
%\end{figure}

%Fig.~\ref{fig:archdiag} shows the role this document plays within the
%IVOA architecture \citep{2010ivoa.rept.1123A}.

%???? and so on, LaTeX as you know and love it. ????



\section{Use cases}

\begin{itemize}
	\item To get basic provenance information in VO output(author, pulication date, article, DOI,...) 
        \item To trace data origin: query, resuources used to compute the result... 
        \item To homogenize the Origin-metadata information in VO output.\\
	Example: Query the Gaia catalogue using VO services (for instance with topcat or any other VO-software). The registry lists Data Center (eg: Gavo, VizieR, ESA) which provides Gaia  tables using TAP. The results returns VOTable having information in the header. However, the information depends of the implementation.\\
	To get citation information require the user to query the providers web sites or to use ADS.
	\item Relevant meta-data for final users to \textbf{cite} resources 
        \item Fill the AAS citation template.\\
         Example : "we searched optical astrometric data of these sources from the Gaia (Gaia Collaboration et al. 2016) Early Data Release 3 (Gaia Collaboration et al. 2021) via the Gaia archive (Gaia Collaboration 2020)."
	\item Relevant meta-data for final users to understand data origin.\\
	Table provided by a Data Center can be a copy of an existing resource. For instance, a table published in a journal or by a Space Agency is also hosted in a Data Center like CDS, GAVO, etc.
	The data curation depends of the Data Center which can add associated data, enrich meta-data (eg: add filter for magnitude) or make a sub-selection of columns.
	\item Give me a bibliography of everything I've used in the workflow"
\end{itemize}

The basics meta-data should contain the data origin (space agency or authors, article references), the data center providing the resource, the date of publication ...

\section{State of the art}

\section{Expected Data Origin}


\appendix
\section{Changes from Previous Versions}

No previous versions yet.  
% these would be subsections "Changes from v. WD-..."
% Use itemize environments.


% NOTE: IVOA recommendations must be cited from docrepo rather than ivoabib
% (REC entries there are for legacy documents only)



\bibliography{ivoatex/ivoabib,ivoatex/docrepo}


\end{document}
