\documentclass[11pt,a4paper]{ivoa}
\input tthdefs
\usepackage{todonotes}

\title{Data Origin in the VO}

% see ivoatexDoc for what group names to use here
\ivoagroup{DCP}

%\author[????URL????]{G.Landais}
\author{G.Landais}
\author{G.Muench}
\author{M.Demleitner}
\author{R.Savalle}
\author{looking for contributors}
%\author{????Fred Offline????}

\editor{G.Landais}

% \previousversion[????URL????]{????Concise Document Label????}
\previousversion{This is the first public release}


\begin{document}
\begin{abstract}
The goal of the document is to make the Data Origin more visible in the query results executed in the Virtual Observatory.
The document lists metadata required to provide sufficient traceability to end-users in order to improve the understanding
of the resultsets and enabling its reuse and its citation.

\textbf{NOTE} in work -  template for a possible IVOA note.


\end{abstract}


\section*{Acknowledgments}
Alberto Accomazzi (ADS), Anne Catherine Raugh (University of Maryland), Rafaele d'Abrusco (CfA), Mihaela Buga (CDS)

\section*{Conformance-related definitions}


\section{Introduction}

Data Origin is required for end users to understand data, for citation and for reusability. The  provenance is cited as a mandatory criterion in the EOSC or in RDA FAIR definition.

The virtual observatory provides an advanced framework to search and consume data provided by data centres or Space Agencies who apply curation in different level.  In this context, Data Origin in output includes metadata from the data producer (author, space agency) and the data centre which hosts the resource.
Depending of the implementation, the users can find the origin information in the data centre web pages (landing pages) or in the Registries of the Virtual Observatory. For citation, ADS (Astrophysics Data System, Nasa) provides citation capabilities with bibtex output. There are no VO standards to communicate this type of information yet. For instance, the origin metadata are not included neither in output format, nor in protocols used to access the data.

A list of basic metadata added reliably findable in a convenient location (e.g.,
in result VOTable or the Registry) will help users to properly cite or
acknowledge the data resources they used.
Tracing Data Origin, from the producer to the final query enables also to report to end users the different agents implied in the data preservation (authors, data centre, space agencies, journal)- especially when data can be subject to a curation  which depends of the different agents.
We propose a list of metadata items which meets the needs of basic provenance
tracking using methods available today for their dissemination.



\section{Use cases}

\begin{itemize}
	\item (Data Origin information) Researchers have data in a VOTable that shows an odd feature. They would now like to talk to the creator of the data to help figure out whether that feature is physics or an artefact.
	
	Requirement: contact information to producers present.
	
	Researchers complete their understanding with Data Origin information easily accessible from the VOTable, regardless of the service which generated the result. For instance, the information contains the authors, the year of publication or related resources like article or the original data URL.
		
	When data provided by the service is derived from external resources, or if the data were performed with an additional curation, the nature and links to external resources are available.
	
	For instance, a table published in a journal or by a Space Agency is also hosted in a data centre like CDS, GAVO, etc. The data curation depends of the data centre which can add associated data, enrich metadata (eg: add filter for magnitude) or make a sub-selection of columns. [an advanced serialisation could be based on DOI vocabulary "isVariantFormOf", "IsDerivedFrom", ...]
	
	\item (Reproducibility) A researcher revisits work they did six months earlier in an ad-hoc fashion and would now like to reproduce it in a more structured fashion. To do that, they need to know, say, which queries against which services, or perhaps which programs, produced the files.
	
	Requirement: have the request parameters and a service identification
	(an ivoid in a narrower VO context, > an access URL beyond that)
	in Data Origin.
	
	\item (Citation) While preparing a publication, a researcher would like to properly cite the software and data that went into their results. They now run a program to extract that information from the digital artefacts going into the publication -- perhaps even in separate parts of citations and acknowledgments.
	
	Requirement: The Data Origin must indicate requests for citation and/or acknowledgment in a machine-readable way, preferably in a way that machines can generate BibTeX for whatever they specify
	
	The information allows the researcher to fill the template citation asked by journals.
	
	Example (American Astronomical Society template):
	
	"we searched optical astrometric data of these sources from the Gaia (Gaia Collaboration et al. 2016) Early Data Release 3 (Gaia Collaboration et al. 2021) via the CDS archive"*
	
	\item (workflow) Give me a bibliography of everything I've used in the workflow"
	The VOTable resulting of a session contains homogenized metadata that can be merged and compared.
	
\end{itemize}

The basics metadata should contain Data Origin (space agency or authors, article references), the data centre providing the resource, the date of publication ...

\section{State of the art}

Neither VOTable \citep{2019ivoa.spec.1021O} nor other IVOA protocols provide any Data Origin information. For instance, the TAP protocol \citep{2019ivoa.spec.0927D} provides tables and columns description in order to query tables. The TAP metadata contains information like unit, type and text information. Authors, publication date or identifiers are not included in the TAP description.

HiPS \citep{2017ivoa.spec.0519F} is a more recent protocol which includes for each Dataset (HiPS) a list of standardized metadata. HiPS metadata include authors, publication year, data centre identifier or licenses.


\subsection{Data Origin in IVOA registry}
The IVOA Registry contains metadata relevant for Data Origin, for instance, authors, publication date, references and alternate identifiers like DOI for each of its resources,
e.g., \xmlel{<altIdentifier>}doi:10.26093/cds/vizier.1355\xmlel{</altIdentifier>} \citep{2018ivoa.spec.0625P}.
It makes this information available through several interfaces, partly
hosted by the data centres themselves, partly provided by a central
infrastructure.
The VO Registry is an open framework without any moderators.
The IVOA hence does not guaranty the resources' sustainability.

The IVOA Registry provides a unique identifier (the ivoid; see \citet{2016ivoa.spec.0523D}) for each resource.  This ivoid has not been considered to be citable in articles, because it is a technical identifier with no provisions for persistence.

Both the Registry's metadata schema and the DataCite
\citep{std:DataCite31} metadata schema have been
derived from Dublin Core.  While the extensions differ in detail, it is not
hard to write mapping between the two metadata schemas.  This can enable
sustainable citation.

\subsection{Data Origin serialisation}
The Provenance \citep{2020ivoa.spec.0411S} and Dataset Data Models can be used to serialize Data Origin.

The Provenance Data Model (ProvDM) is based on Entities, Agents and Activities which are defined in the W3C Provenance. The model is dedicated to serialize workflow,
it is a recursive model that can be serialized in JSON or RDF.

The ProvDM serialization needs mivot to be serialized in VOTable. But the recursive model and the metadata serialization which is subject to the producer implementation, are obstacles for its reusability.


The "Last-Step-Provenance" is a Provenance extension (not a standard yet) which gather a list of metadata which matches with Data Origin. Its output is not recursive and could be easily serialized in a table.


DatasetDM provides a table description which includes Data Origin information like authors, date of publication, links to bibliography, etc.
DatasetDM is not a standard yet.

\subsection{DALI}
DALI \citep{2017ivoa.spec.0517D} is a basis for several VO protocols, used for instance in TAP. It includes services to  check the availability and capabilities of a service. It includes also information in VOTable as the request STATUS or the REQUEST query.

It already defines bespoke names for \xmlel{INFO} elements used to convey additional metadata, in particular \emph{citation}.  In a sense, this specification is an extension of the mechanism defined in DALI.

\section{Expected Data Origin}

List of expected metadata:

\begin{itemize}
\item Author -- name or ORCID of the Data producer (human)
\item Organization--  name, ROR or URL of the Data producer (organization)
\item Editor -- name or URL  of the editor (when data are attached to a publication)
\item Journal -- name or URL of a journal (or similar aggregation of publications) holding a publication on data that was used to build the current resource
\item data centre -- name, ROR or URL of the data centre who hosts the data
\item Contact -- data centre email
\item Resource Identifier -- ivoid of resource(s) hosted by the service which provides the result
\item Resource citation -- DOI, bibcode of resource(s) hosted by the data centre which returns the result
\item Original resource identifier -- remote/original resource which was used to build the result
\item Publication date -- publication date in the data centre
\item Original Publication date --  publication date of the original resource
\item Curation level -- curation level in the data centre
\item Operation -- Operation as cutout, add-values executed on Data-center on the original data
\item License -- (original) licenses - an SPDX\footnote{\url{https://spdx.dev/}}
URI is preferred
\item Data version -- version or release
\item Access protocol --  e.g., TAP, SCS, \dots
\item Query -- In protocols like TAP, the query passed in by the user as
as single string
\item Version -- version or date of Data-center software
\item ... (\textbf{to complete?})
\end{itemize}


\subsection{Condition for citation}
\todo{I don't know if it is the place to talk about that??}

The DOI is the privileged persistent identifier to cite resources.\\

%Data Citation requires a sustainable URL which is not guarantied in IVOA resources.
%Unlike ivoid, the DOI guaranties a sustainable URL and should be used for citation. \\
Data citation requires a persistent identifier and a sustainable URL. 
Both are guaranteed by DOI, but resource provided with an ivoid (the IVOA identifier) 
is not guaranteed to be sustainable.\\

Bibtex requires curation that needs metadata like identifier, authors, title, publisher and date of publication.
ADS (NASA) provides a citation capability for its indexed resources. This curation quality has to be privileged or may be took as example for any data providers and users.

For instance, DOI providers like Datacite, provides a bibtex capability. The bibtex quality depends of the DOI metadata filled by the Data producers and publishers.\\

The IVOA registry which contains metadata for any resources could be used to get the expected quality for citation if the following conditions are met:
\begin{itemize}
\item the registry resource includes a persistent identifier (DOI)
\item the registry resource includes the metadata which meets the bibtex requirements
\end{itemize}

\begin{table}
\begin{tabular}{|l|l|l|l|}  \hline
\textbf{metadata} & \textbf{Description} & \textbf{Level} & \textbf{Dublin Core}\\ \hline
ivoid             & ivoid identifier to link registry & M &  \\ \hline
publisher         & data centre that provides the VOTable & M & publisher\\ \hline
version           & Software version  & & \\ \hline
service\_protocol & Protcol access with version & R& \\ \hline
request           & Request url &  R& \\ \hline
request\_post     & (POST Request) POST arguments &  & \\ \hline
request\_date     & Query execution date & R&\\ \hline	
contact           & email or URL to contact publisher & & \\ \hline	
landing\_page     & Dataset landing page & & \\ \hline
\end{tabular}
\caption{\xmlel{INFO} names available for specifying the query that
generated a VOTable}
\label{tab:query-names}
\end{table}

\begin{table}
\begin{tabular}{|l|p{5cm}|l|l|}  \hline
\textbf{metadata} & \textbf{Description} & \textbf{Level} & \textbf{Dublin Core}\\ \hline
publication\_id    & Dataset identifier that can be used for citation& M  & identifier\\ \hline
curation\_level    & Controled vocabulary
                   (IVOA rdf, content\_level) &  &  \\ \hline
resource\_version  & Dataset version or last release & R & \\ \hline
rights             & (*) Licence URI & R & rights\\ \hline
rights\_type       & (*) Licence type (eg: CC-by, CC-0, private, public) &  &  \\ \hline
copyrights         & Copyright text &  & \\ \hline
creator            & The person or organization primarily responsible for creating the
                     intellectual content of the resource.  For example, authors in the
                     case of written documents, artists, photographers, or illustrators in
                     the case of visual resources. & R & creator\\ \hline
editor             & Editor name &  & \\ \hline
relation\_type     & An identifier of a second resource and its relationship to the
                     present resource.
                     Controlled vocabulary (\textbf{TODO}) & & relation\\ \hline
related\_resource  & Information about a second resource from which the present resource
                     is derived. The source is an identifier that can be prefixed with the identifier type: eg: bibcode:, doi:, ror: &  & source\\ \hline
publication\_date  & Date of publication (format ISO 8601) &  R &  \\ \hline
resource\_date     & Date of the original publication (format ISO 8601) & R & date\\ \hline
\multicolumn{4}{p{\textwidth}}{\footnotesize(*) The right element	accepts free
text. However,	it is preferable to provide	a Machine-readable
Licence.}\\
\end{tabular}
\caption{\xmlel{INFO} names available for specifying information
related to the origin of the data set(s) a VOTable was generated from}
\label{tab:origin-names}
\end{table}


\section{Data Origin in VOTable}

The metadata listed bellow combines terms from DALI (REC-DALI-1.1), Dublin Core (DC: Dublin core, RFC 2413) and extensions in order to reproduce and to provide Data Origin information to end users.

\subsection{Query information}
Table~\ref{tab:query-names} lists the metadata items defined here to
convey query-related information in Data Origin.

These pieces of information enable linking Registry records to the
current result and, to some extent, reproducing the query executed. For
queries on evolving datasets, the version or the date must complete the
information.

\subsection{Dataset Origin}
Dataset origin complements the query-related information to improve the
understandability of the underlying data. This information is intended
for end users.  If the resource is also described in the Registry, care
must be taken that in-response metadata remains in sync with metadata
available there.

Table~\ref{tab:origin-names} lists the origin-related metadata items
defined here.


\subsection{VOTable serialization}

In this document, we focused on a basic serialization that allows description implying individual tables.
This output is adapted for protocol like the Simple Conesearch.

The basic serialization uses INFO tags to populate Data Origin (see the example of a ConeSearch result in appendix  \ref{appendixA}).
INFO tags are allowed in VOTable under \xmlel{VOTABLE} or in \xmlel{RESOURCE} elements.
Thus, it becomes possible to annotate a collection of TABLE which are in different resources.


Complex query (for instance, a join in a TAP query) needs an advanced output serialization to gather metadata by resource.
Mechanisms to manage this requirement are being developed in IVOA (mivot).
This output is not treated in the current version of the document.



%\subsubsection{Complex output involving several tables (eg: TAP query, ObsCore result)}
%Dataset-origin depends on each table used for the output. Datamodels like Last-step -Provenance or DatasetDM allows to gather the metadata.
%
%DatasetDM Example:
%
%Dataset-origin depends on each table used for the output. Datamodels like Last-step -Provenance or DatasetDM allows to gather the metadata.
%Both could be serialized using mivot in VOTable.
%
%DatasetDM Example:
%
%\begin{tabular}{|l|p{5cm}|l|}  \hline
%\textbf{meta-data} &\textbf{Description} & \textbf{Level} \\ \hline
%dataset:productType & & \\ \hline
%dataset:productSubType & controled vocabulary & \\ \hline
%dataset:DataID.datasetDID & dataset ivoid & M \\ \hline
%dataset:DataID.title & dataset title & \\ \hline
%dataset:DataID.creationType & type of resource & \\ \hline
%dataset:DataID.date & Publication date of original dataset/article & \\ \hline
%dataset:Party.name  & (first)Author & \\ \hline
%dataset:Curation.publisherDID & data-center identifier (ivoid) & M \\ \hline
%dataset:Curation.rights	rights & text & \\ \hline
%dataset:Curation.releaseDate   & Data-center publication date & M \\ \hline
%party.Organisation.email & Data-center contact & \\ \hline
%dataset:Curation.doi     & Dataset DOI & \\ \hline
%dataset:Curation.bibcode & Dataset bibcode & \\ \hline
%\end{tabular}
%
%\textbf{Question: do we have to provide an example }

\section{Data Origin in Registry}
The ivo-id, now available in VOTable, allows to query the resource metadata which are in the VO registry.\\


Expected metadata (VOResource) with their equivalent in Datacite schema (version 4.4) to provide Data Origin in the registry\\

\begin{tabular}{|p{3cm}|p{4cm}|p{1cm}|p{5cm}|} \hline
\textbf{VOResource} & \textbf{DataCite} & \textbf{Level} & \textbf{Explain} \\ \hline
identifier    &Identifier (1) &M & ivoid of resource(s) hosted by the service\\ \hline
title         &Title (3) &M  & resource title\\ \hline
shortName     &&& Resource short name\\ \hline
altIdentifier & AlternateIdentifier (11)& R &
              Alternate identifier accepts bibcode, DOI or URL. DOI should be privileged to facilitate citation and link with DataCite or Crossref..eg: DOI \\ \hline
\multicolumn{4}{l}{(*) terms allowing name and Orcid (AltIdentifier in VOResurce)} \\
\end{tabular}

\begin{tabular}{|p{3cm}|p{4cm}|p{1cm}|p{5cm}|} \hline
\multicolumn{4}{|l|}{\textbf{curation}} \\ \hline
.publisher     & Publisher (4) & M &publisher (*)\\ \hline
.creator       & Creator (2) & M & author(s) (*)\\ \hline
.contributor   & Contributor & & contributor(s) (*)\\ \hline
.date [Created]& Dates [created] (8)& M & creation date (in data centre)\\ \hline
.date [Updated]& Dates [updated] (8)& M & last modification\\ \hline
  ?            & PublicationYear (5) & & publication year in data centre\\ \hline
.version       & Version (15) & R &\\ \hline
.contact       & &&\\ \hline
\end{tabular}

\begin{tabular}{|p{3cm}|p{4cm}|p{1cm}|p{5cm}|} \hline
\multicolumn{4}{|l|}{\textbf{content} } \\ \hline
.source        & RelatedIdentifier (12) (type=bibcode, relationType=IsSupplementTo) & R & bibcode\\ \hline
.referenceURL  & & R & landing page\\ \hline
.type          & ResourceType (10)& & Resource type (catalog, etc)\\ \hline
.description   & Description (17)& & abstract\\ \hline
.contentLevel  & & &\\ \hline
.relationShip  & RelatedIdentifiers (12) & R &link to remote resource (Recommended to link Original data centre) \\ \hline
..relationshipType & relationType (12.b) & &\\ \hline
..relatedResource  & RelatedIdentifier (12) & R &\\ \hline
\end{tabular}

\begin{tabular}{|p{3cm}|p{4cm}|p{1cm}|p{5cm}|} \hline
\multicolumn{4}{|l|}{\textbf{rights}} \\ \hline
rights   & Rights (16)& R& licence
          The right element accepts free text. However, it is preferable to provide a machine-readable License. See the list https://spdx.org/licenses/.
         \\ \hline
.URI     & rightsURI & R& licence URL\\ \hline
         & rightsIdentifier & & standard licence name .ex CC-by.
         Copyright is accepted by FAIR principle. But copyright is only a link to the data producer. It gives the contact point to any users who would like to use data. Copyright is more simple to implement for data-center that provides a copy of original resource, but its use is not well integrated in an interoperable workflow.
         \\ \hline

\end{tabular}\\




\textbf{Examples}

Example of rights serialization:
\begin{verbatim}
<rights rightsURI="https://spdx.org/licenses/CC-BY-4.0.html">
  Creative Commons Attribution 4.0
</right>
\end{verbatim}


Example or relation ship :
Cite the original dataset using "source" (to link a bibliographic reference) or "relatedIdentifier" (to link a dataset)

e.g.:
\begin{verbatim}
<relationship>
  <relationshipType>Cites</relationshipType>
  <relatedResource>doi: 10.5270/esa-qa4lep3 : Gaia DR3 ESA</relatedResource>
</relationship>
\end{verbatim}

Example of Creator:
\begin{verbatim}
<creator>
	<name>Bryson S.</name>
	<altIdentifier>orcid:0000-0003-0081-1797<altIdentifier>
</creator>
\end{verbatim}

% may be in an other note?
%
% UPDATE - licenses : type, uri as Datacite
% ADD - copyrights
% ADD - akcnowledgement
% ADD - curation text.. added values, column selection...., extract....
% list of Data Origin metadata  in registry

\section{Acknowledgment template}

Example of acknowledgment that can be extract from the metadata:\\

\textbf{Template}:\\
We extract data published in <related\_resource> (<creator>, <resource\_date>),
via <publisher> services (ivoa resource=<ivoid>, <publication\_date>)
using <service\_protocol> (version <version>, executed at <request\_date>\\


\textbf{Example}:\\
We extract data published in bibcode:2021AJ....161...36B (Bryson S., 2021),
via CDS services (ivoa resource=ivo://cds.vizier/j/aj/161/36, 2021-03-16)
using Simple Cone Search 1.03 (version 7.294, executed at 2022-10-30)



\appendix
\section{Appendix, Cone search serialization}\label{appendixA}
Simple Conesearch with its VOTable serialization. Data Origin are specified using  INFO.
\begin{verbatim}
<VOTABLE version="1.1" xmlns:xsi="http://www.w3.org/2001/XMLSchema-instance"
    xmlns="http://www.ivoa.net/xml/VOTable/v1.1" xsi:schemaLocation=...>

  <INFO name="protocol" value="Simple Cone Search 1.03"/><!-- URL or text ? -->
  <INFO name="request_date" value="2022-10-30T12:08:00"/>
  <INFO name="request"
   value="https://vizier.cds.unistra.fr/viz-bin/conesearch/J/AJ/161/36/table8?RA=285.4%26DEC=39.3%26SR=1"/>
  <INFO name="contact" value="cds-question@unistra.fr"/>
  <INFO name="version" value="7.294"/>


  <RESOURCE ID="yCat_51610036" name="J/AJ/161/36">
    <DESCRIPTION>117 exoplanets in habitable zone with Kepler DR25</DESCRIPTION>

    <INFO name="ivoid" value="ivo://cds.vizier/j/aj/161/36"/>
    <INFO name="publisher" value="doi:10.26093/cds/vizier.51610036"/>
    <INFO name="landing_page"
          value="https://cdsarc.cds.unistra.fr/viz-bin/cat/J/AJ/161/36"/>

    <!-- Extra information from Data Origin (basic info) -->
    <INFO name="publication_id" value="doi:10.26093/cds/vizier.51610036"/>
    <INFO name="curation_level" value="Research"/>
    <INFO name="resource_version" value="2022-10-07"/>
    <INFO name="rights" value="https://cds.unistra.fr/vizier-org/licences_vizier.html"/>
    <INFO name="creator" value="Bryson S."/ ><!-- ORCID ? -->
    <INFO name="related_resource" value="2021AJ....161...36B"/>
    <INFO name="editor" value="Astronomical Journal"/>
    <INFO name="publication_date" value="2021-03-16"/>
    <INFO name="resource_date" value="2021"/>
    ....
    <TABLE>  .... </TABLE>
  </RESOURCE>
</VOTABLE>
\end{verbatim}

\section{Appendix, Changes from Previous Versions}

No previous versions yet.
% these would be subsections "Changes from v. WD-..."
% Use itemize environments.


\bibliography{ivoatex/ivoabib,ivoatex/docrepo}


\end{document}
