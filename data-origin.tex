\documentclass[11pt,a4paper]{ivoa}
\input tthdefs
\usepackage[textsize=small]{todonotes}

\title{Data Origin in the VO}

% see ivoatexDoc for what group names to use here
\ivoagroup{DCP}

%\author[????URL????]{G. Landais}
\author{G. Landais}
\author{G. Muench}
\author{M. Demleitner}
\author{R. Savalle}
%\author{looking for contributors}
%\author{????Fred Offline????}

\editor{G. Landais}

% \previousversion[????URL????]{????Concise Document Label????}
\previousversion{This is the first public release}


\begin{document}
\begin{abstract}
The goal of the document is to make Data Origin more visible in query results obtaines from the Virtual Observatory.
The document lists metadata required to provide sufficient traceability to end-users in order to improve the understanding
of the result sets and facilitating its reuse and its citation.
\end{abstract}


\section*{Acknowledgments}
Alberto Accomazzi (ADS), Anne Catherine Raugh (University of Maryland), Rafaele d'Abrusco (CfA), Mihaela Buga (CDS)


\section{Introduction}

Data Origin, in the sense of shallow metadata on who generated a piece of data how, when, and why, is required for end users to understand data, to properly cite it and to confidently re-use it.  Provenance -- somewhat wider than our Data Origin -- is cited as a mandatory criterion in the EOSC or in the RDA FAIR definition.

The Virtual Observatory (VO) provides an advanced framework for searching and consuming data provided by data centres or space agencies that apply curation at different levels.  In this context, Data Origin includes metadata from the data producer (author, space agency) and the data centre which hosts the resource.
Depending on the implementation, users can find the origin information in data centre web pages (landing pages) or in the VO Registry. For citation, the NASA's ADS (Astrophysics Data System) provides citation capabilities with BibTeX output.

 There are no VO standards to communicate this type of information yet. In particular, Data Origin is not part of the output formats defined for the various protocols defined by the VO.
A well-defined set of basic metadata reliably findable in a convenient location (typcially, a result VOTable) will help users to properly cite or
acknowledge the data resources they used.
Tracing Data Origin from the producer to the final consumer also enables reports on the different agents involved in the curation and preservation of the source data (authors, data centre, space agencies, journal).

The remaining note is organised into a better definition of our use
cases in sect.~\ref{sect:usecases}, followed by a review of existing
means of communicating Data Origin in the VO in
sect.~\ref{sect:stateoftheart}.  In sect.~\ref{sect:expecteddataorigin},
we define the pieces of metadata making up Data Origin, while
sect.~\ref{sect:invotable} gives their VOTable serialisation.  Finally,
a few words on the relationship to the VO Registry are provided in
sect.~\ref{sect:registry}, and an illustration of using Data Origin for
machine-generated acknowledgements is given in
sect.~\ref{sect:template}.


\section{Use cases}
\label{sect:usecases}

\begin{itemize}
	\item (Data Origin information) Researchers have data in a VOTable that shows an odd feature. They would now like to talk to the creator of the data to help figure out whether that feature is physics or an artefact.
	
	Requirement: contact information to producers present.
	
	Researchers complete their understanding with Data Origin information easily accessible from the VOTable, regardless of the service which generated the result. For instance, the information contains the authors, the year of publication or related resources like article or the original data URL.
		
	When data provided by the service is derived from external resources,
	or if the some additional curation was applied, Data Origin allows
	data consumers to understand such processes.
	
	For instance, a table published in a journal or by a space agency is also hosted in a data centre like CDS, GAVO, etc. The data curation depends of the data centre, which can add associated data, enrich metadata (e.g., add a filter for magnitude) or only publish a sub-selection of the columns in the original source.
	
	\item (Reproducibility) A researcher revisits work they did six months earlier in an ad-hoc fashion and would now like to reproduce it more carefully. To do that, they need to know, say, which queries against which services, or perhaps which programs, produced the on-disk artefacts.
	
	Requirement: have the request parameters and a service identification
	(an ivoid in a narrower VO context, an access URL beyond that)
	in Data Origin.
	
	\item (Citation) While preparing a publication, a researcher would like to properly cite the software and data that went into their results. They now run a program to extract that information from the digital artefacts going into the publication -- perhaps even in separate parts of citations and acknowledgments.
	
	Requirement: The Data Origin must indicate requests for citation and/or acknowledgment in a machine-readable way, preferably in a way that machines can generate BibTeX for whatever they specify.
	
	The information provided should help researchers fill the citation templates asked by journals, such as (using a template of the American Astronomical Society): ``we searched optical astrometric data of these sources from the Gaia (Gaia Collaboration et al. 2016) Early Data Release 3 (Gaia Collaboration et al. 2021) via the CDS archive''.
	
	\item (workflow) Researches would like to obtain a bibliography of everything I've used in a workflow leading up to some result.
	With the Data Origin proposed here, the VOTables accumulating in a session contain fairly homogeneous metadata that can be merged and compared by machines.
	
\end{itemize}

\section{State of the Art}
\label{sect:stateoftheart}

Neither VOTable \citep{2019ivoa.spec.1021O} nor the common DAL protocols provide any Data Origin information. For instance, while the TAP protocol \citep{2019ivoa.spec.0927D} does require services to publish units, types, and descriptions of table columns in well-defined formats, no provision is made to machine-readably declare authors, a publication date, or origin identifiers.

HiPS \citep{2017ivoa.spec.0519F} is a more recent protocol which includes for each Dataset (HiPS) a list of standardized metadata. HiPS metadata include authors, publication year, data centre identifier or licenses.


\subsection{Data Origin in IVOA registry}
The IVOA Registry contains metadata relevant for Data Origin, for instance, authors, a publication date, a source reference, or identifiers like ivoids or DOIs \citep{2018ivoa.spec.0625P}.
It makes this information available through several interfaces, partly
hosted by the data centres themselves \citep{2017ivoa.spec.0524G}, partly on common
infrastructures.
Note that the VO Registry is an open framework without any moderators, and
there are no formal guarantees on any
resource's sustainability.

The IVOA Registry's primary key is the IVOA Identifier or ivoid  \citep{2016ivoa.spec.0523D}, giving each resource registered a globally unique URI with the scheme \emph{ivo}.  The ivoid has not been considered to be a means of citation in articles, because it is a technical identifier with no provisions for persistence.

Both the Registry's metadata schema and the DataCite
\citep{std:DataCite40} metadata schema have been
derived from Dublin Core \citep{std:DUBLINCORE}.  While the extensions differ in detail, it is not
hard to write mapping from VOResrouce to DataCite, which facilitates
obtaining persitent DOIs for VO resources, thus enabling robust citation practices for registered VO resources.

\subsection{Data Origin serialisation}
The Provenance Data Model ProvDM \citep{2020ivoa.spec.0411S} could be used to serialize Data Origin.
It is based on Entities, Agents and Activities as defined by the W3C Provenance Model and can serialize complex workflows.  As ProvDM is written in VO-DML, provenance written in it would be serialised into VOTables using  mivot \todo{citation}.  But the recursive nature of that model and the flexible metadata serialization are clear obstacles for wide adoption and interoperable consumption at this point.

``Last-Step-Provenance'' is a proposed Provenance restriction which gathers a list of metadata which matches with Data Origin. Its output is not recursive and could be easily serialized in a table.

\subsection{DALI}
DALI \citep{2017ivoa.spec.0517D} lays down basic conventions applicable all VO data access protocols such as TAP.
It already defines bespoke names for \xmlel{INFO} elements used to convey additional metadata, in particular \emph{citation}.  In a sense, this specification is an extension of the mechanism defined in DALI.

\section{Expected Data Origin}
\label{sect:expecteddataorigin}

The following pieces of metadata are defined by version \ivoaDocversion~of this
specification:\todo{Is it really useful to have this separate from current section 5?  Doesn't that just mean repeating everything?}

\begin{description}
\item [Author] Name or ORCID of the Data producer (human)
\item [Organization]  Name, ROR or URL of the Data producer (organization)
\item [Editor] Name or URL  of the editor (when data are attached to a publication)
\item [Journal] Name or URL of a journal (or similar aggregation of publications) holding a publication on data that was used to build the current resource
\item [data centre] Name, ROR or URL of the data centre who hosts the data
\item [Contact] Data centre email
\item [Resource Identifier] Ivoid of resource(s) hosted by the service which provides the result
\item [Resource citation] DOI, bibcode of resource(s) hosted by the data centre which returns the result
\item [Original resource identifier] Remote/original resource which was used to build the result
\item [Publication date] Publication date in the data centre
\item [Original Publication date]  Publication date of the original resource
\item [Curation level] Curation level in the data centre
\item [Operation] Operation as cutout, add-values executed on Data-center on the original data
\item [License] (original) Licenses - an SPDX\footnote{\url{https://spdx.dev/}}
URI is preferred
\item [Data version] Version or release
\item [Access protocol]  e.g., TAP, SCS, \dots\todo{If this is supposed to be machine-readable, we need to be more specific.  Also, see SSAP SERVICE\_PROTOCOL and INFO/standardID.}
\item [Query] In protocols like TAP, the query passed in by the user as
as single string
\item [Version] Version or date of data centre software\todo{Version is
far too generic a name for that -- everyone would expect that to hold
the data version.  See also the operational identification note}
%\item ... (\textbf{to complete?})
\end{description}


\subsection{Condition for citation}
%\todo{I don't know if it is the place to talk about that??}

The DOI is the privileged persistent identifier to cite resources.\\

%Data Citation requires a sustainable URL which is not guarantied in IVOA resources.
%Unlike ivoid, the DOI guaranties a sustainable URL and should be used for citation. \\
Data citation requires a persistent identifier and a sustainable URL.
Both are guaranteed by DOI, but a resource provided with an ivoid (the IVOA identifier)
is not guaranteed to be sustainable.\todo{I'd still drop the whole
section -- the problem is complex, and a discussion IMHO doesn't belong
here.}

Bibtex requires curation that needs metadata like identifier, authors, title, publisher and date of publication.
ADS (NASA) provides a citation capability for its indexed resources. This curation quality has to be privileged or may be took as example for any data providers and users.

For instance, DOI providers like Datacite, provides a bibtex capability. The bibtex quality depends of the DOI metadata filled by the Data producers and publishers.\\

The IVOA registry which contains metadata for any resources could be used to get the expected quality for citation if the following conditions are met:
\begin{itemize}
\item the registry resource includes a persistent identifier (DOI)
\item the registry resource includes the metadata which meets the bibtex requirements
\end{itemize}

\begin{table}
\raggedright
\begin{tabular}{|l|p{7cm}|l|l|}  \hline
\textbf{INFO name} & \textbf{Description} & \textbf{Level} & \textbf{Dublin Core}\\ \hline
ivoid             & IVOA identifier of the originating resource & M &  \\ \hline
publisher         & Data centre of origin & M & publisher\\ \hline
version           & Version of the generating software (*) & & \\ \hline
service\_protocol & Access protocol through which the data was retrieved & R& \\ \hline
request           & Full request URL including GET parameters &  R& \\ \hline
request\_post     & (POST Request) POST arguments &  & \\ \hline
request\_date     & Date and time of VOTable generation, DALI-style ISO & R&\\ \hline	
contact           & Contact e-mail address of the publisher (or equivalent URL) & & \\ \hline	
landing\_page     & Dataset landing page & & \\ \hline
\multicolumn{4}{l}{\footnotesize(*) free text, standardised semantic versioning encouraged, see \url{https://semver.org/}} \\
\end{tabular}
\caption{\xmlel{INFO} names available for specifying the query that
generated a VOTable}
\label{tab:query-names}
\end{table}

\begin{table}
\begin{tabular}{|l|p{5cm}|l|l|}  \hline
\textbf{metadata} & \textbf{Description} & \textbf{Level} & \textbf{Dublin Core}\\ \hline
publication\_id    & Dataset identifier that can be used for citation& M  & identifier\\ \hline
curation\_level    & Controled vocabulary
                   (IVOA rdf, content\_level) &  &  \\ \hline
resource\_version  & Dataset version or last release & R & \\ \hline
rights             & (*) Licence URI & R & rights\\ \hline
rights\_type       & (*) Licence type (eg: CC-by, CC-0, private, public) &  &  \\ \hline
copyrights         & Copyright text &  & \\ \hline
creator            & The person or organization primarily responsible for creating the
                     intellectual content of the resource.  For example, authors in the
                     case of written documents, artists, photographers, or illustrators in
                     the case of visual resources. & R & creator\\ \hline
editor             & Editor name &  & \\ \hline
relation\_type     & An identifier of a second resource and its relationship to the
                     present resource.
                     Controlled vocabulary (**)& & relation\\ \hline
related\_resource  & Information about a second resource from which the present resource
                     is derived. The source is an identifier that can be prefixed with the identifier type: eg: bibcode:, doi:, ror: &  & source\\ \hline
publication\_date  & Date of publication (format ISO 8601) &  R &  \\ \hline
resource\_date     & Date of the original publication (format ISO 8601) & R & date\\ \hline
\multicolumn{4}{p{\textwidth}}{\footnotesize(*) The right element	accepts free
text. However,	it is preferable to provide	a Machine-readable
Licence.}\\
\multicolumn{4}{p{\textwidth}}{\footnotesize(**) \url{https://www.ivoa.net/rdf/voresource/relationship_type/}}
\end{tabular}
\caption{\xmlel{INFO} names available for specifying information
related to the origin of the data set(s) a VOTable was generated from}
\label{tab:origin-names}
\end{table}


\section{Data Origin in VOTable}
\label{sect:invotable}

The metadata listed below combines terms from DALI version 1.1, Dublin Core (DC: Dublin core, RFC 2413), and local extensions available to convey standardised Data Origin information to end users.

\subsection{Query Information}
Table~\ref{tab:query-names} lists the metadata items defined here to
convey query-related information in Data Origin.

Comments on items there \todo{remove this paragraph after looking at it.}
: (a) on version, see below.  I'd just cite the operational identification note, but again, this really needs to be renamed.  (b) service\_protocol: as I say somewhere else, we need to unify this with the standardId proposal that's been around for a time now, too.  (c) request\_post: I don't think that's a good idea; these can come in both application/multipart and form-encoded, and they can be large.  You could possibly say post\_payload and say ``don't dump megabytes of stuff here'', but I'm not sure that's a good plan. (d) For landing\_page, you'd have to explain the relationship to the VOResource reference URL -- why not call it reference URL to begin with?

These pieces of information enable linking Registry records to the
current result and, to some extent, reproducing the query executed. For
queries on evolving datasets, the version or the date must complete the
information.

\subsection{Dataset Origin}
Dataset origin complements the query-related information to improve the
understandability of the underlying data. This information is intended
for end users.  If the resource is also described in the Registry, care
must be taken that in-response metadata remains in sync with metadata
available there.


Table~\ref{tab:origin-names} lists the origin-related metadata items
defined here.

Comments on items there \todo{remove this paragraph after looking at it.}
(a) I think you want publication\_id to be what source is in VOResource.  If so, you should not call it ``Dataset identifier'' -- what about ``bibliographic identifier''?
(b)  curation\_level quite certainly is not content\_level, is it?  What would research/amateur/general have to do with curation?
(c) rights says ``License URI'' only to say ``it can be free text'' in the table note.  That's not good.  Either say it's free text for human consumption or a URI for machine consumption (where I'm not sure there's a major case for this to be machine-readable; if there is one, I'd say you should copy the rights vs. rightsURI split from VOResource and then just drop copyrights (I'm not too fond of that plural anyway).
(d) In creator, you write ``\textit{The} person''.  Rather make that ``A person'', because there will usually be more than one of these.
(e) On editor, I've already written something a while ago, and I'd still maintain that if you keep it in you have to say the editor of what.  Datasets don't usually have editors for all I can see.
(f) For relation\_type and related\_resource, I think you have to require there's only one of these (otherwise there's no way a client can reconstruct what id belongs to which relationship type.
(g) x\_date: say ``DALI-style ISO timestamp'' rather than ISO 8601.  You wouldn't believe how many junky date formats ISO 8601 admits.  And citing ISO standards sucks anyway as long as ISO tries to sell their wares for a huge fortune.

\subsection{VOTable Serialization}

The basic serialization uses INFO tags to populate Data Origin.  For an example, see the result of a cone search \citep{2008ivoa.specQ0222P} query in appendix  \ref{appendixA}.

INFO tags are allowed in VOTable under \xmlel{VOTABLE} or, preferably, in \xmlel{RESOURCE} elements.
The latter option facilitates joining VOTables without losing each individual dataset's Data Origin.

Complex queries (for instance, cross-table a joins in ADQL query) may need a more complex, hierarchical metadata model for full Data Origin description and a correspondingly more complex serialisation.  Data providers are encouraged to provide best-effort Data Origin in these cases (e.g., by including multiple parallel license or citation \xmlel{INFO}-s.



%\subsubsection{Complex output involving several tables (eg: TAP query, ObsCore result)}
%Dataset-origin depends on each table used for the output. Datamodels like Last-step -Provenance or DatasetDM allows to gather the metadata.
%
%DatasetDM Example:
%
%Dataset-origin depends on each table used for the output. Datamodels like Last-step -Provenance or DatasetDM allows to gather the metadata.
%Both could be serialized using mivot in VOTable.
%
%DatasetDM Example:
%
%\begin{tabular}{|l|p{5cm}|l|}  \hline
%\textbf{meta-data} &\textbf{Description} & \textbf{Level} \\ \hline
%dataset:productType & & \\ \hline
%dataset:productSubType & controled vocabulary & \\ \hline
%dataset:DataID.datasetDID & dataset ivoid & M \\ \hline
%dataset:DataID.title & dataset title & \\ \hline
%dataset:DataID.creationType & type of resource & \\ \hline
%dataset:DataID.date & Publication date of original dataset/article & \\ \hline
%dataset:Party.name  & (first)Author & \\ \hline
%dataset:Curation.publisherDID & data-center identifier (ivoid) & M \\ \hline
%dataset:Curation.rights	rights & text & \\ \hline
%dataset:Curation.releaseDate   & Data-center publication date & M \\ \hline
%party.Organisation.email & Data-center contact & \\ \hline
%dataset:Curation.doi     & Dataset DOI & \\ \hline
%dataset:Curation.bibcode & Dataset bibcode & \\ \hline
%\end{tabular}
%
%\textbf{Question: do we have to provide an example }

\section{Data Origin and the Registry}
\label{sect:registry}

The IVOA identifier of the originating resource as available from the
\emph{ivoid} \xmlel{INFO} allows clients to query additional resource metadata
from the VO Registry.

The following tables\todo{again, I think a definition list or at least something with fewer rulers would improve presentation} show the relationship between VOResource elements and their equivalent in the Datacite schema (version 4.4) as far as Data Origin is concerned.\todo{I have not reviewed this for now. Let's chat about it, perhaps.} Also note that an XSLT schema for mapping VOResource into DataCite is available\footnote{\url{https://github.com/ivoa/vor-doi}}.

\begin{tabular}{|p{3cm}|p{4cm}|p{1cm}|p{5cm}|} \hline
\textbf{VOResource} & \textbf{DataCite} & \textbf{Level} & \textbf{Explain} \\ \hline
identifier    &Identifier (1) &M & ivoid of resource(s) hosted by the service\\ \hline
title         &Title (3) &M  & resource title\\ \hline
shortName     &&& Resource short name\\ \hline
altIdentifier & AlternateIdentifier (11)& R &
              Alternate identifier accepts bibcode, DOI or URL. DOI should be privileged to facilitate citation and link with DataCite or Crossref..eg: DOI \\ \hline
\end{tabular}

\begin{tabular}{|p{3cm}|p{4cm}|p{1cm}|p{5cm}|} \hline
\multicolumn{4}{|l|}{\textbf{curation}} \\ \hline
.publisher     & Publisher (4) & M &publisher (*)\\ \hline
.creator       & Creator (2) & M & author(s) (*)\\ \hline
.contributor   & Contributor & & contributor(s) (*)\\ \hline
.date [Created]& Dates [created] (8)& M & creation date (in data centre)\\ \hline
.date [Updated]& Dates [updated] (8)& M & last modification\\ \hline
  ?            & PublicationYear (5) & & publication year in data centre\\ \hline
.version       & Version (15) & R &\\ \hline
.contact       & &&\\ \hline
\multicolumn{4}{l}{\footnotesize(*) terms allowing name and Orcid (AltIdentifier in VOResurce)} \\
\end{tabular}

\begin{tabular}{|p{3cm}|p{4cm}|p{1cm}|p{5cm}|} \hline
\multicolumn{4}{|l|}{\textbf{content} } \\ \hline
.source        & RelatedIdentifier (12) (type=bibcode, relationType=IsSupplementTo) & R & bibcode\\ \hline
.referenceURL  & & R & landing page\\ \hline
.type          & ResourceType (10)& & Resource type (catalog, etc)\\ \hline
.description   & Description (17)& & abstract\\ \hline
.contentLevel  & & &\\ \hline
.relationShip  & RelatedIdentifiers (12) & R &link to remote resource (Recommended to link Original data centre) \\ \hline
..relationshipType & relationType (12.b) & &\\ \hline
..relatedResource  & RelatedIdentifier (12) & R &\\ \hline
\end{tabular}

\begin{tabular}{|p{3cm}|p{4cm}|p{1cm}|p{5cm}|} \hline
\multicolumn{4}{|l|}{\textbf{rights}} \\ \hline
rights   & Rights (16)& R& license
          The right element accepts free text. However, it is preferable to provide a machine-readable License (*)
         \\ \hline
.URI     & rightsURI & R& license URL\\ \hline
         & rightsIdentifier & & standard license name .ex CC-by.
         Copyright is accepted by FAIR principle. But copyright is only a link to the data producer. It gives the contact point to any users who would like to use data. Copyright is more simple to implement for data-center that provides a copy of original resource, but its use is not well integrated in an interoperable workflow.
         \\ \hline
\multicolumn{4}{p{\textwidth}}{\footnotesize(*) See SPDX list \url{https://spdx.org/licenses/} or Creative Commons licenses \url{https://creativecommons.org}}
\end{tabular}\\

To obtain the respective pieces of metadata from the Registry, clients would typically use RegTAP \citep{2019ivoa.spec.1011D} on a full searchable registry; see \url{https://rofr.ivoa.net} for a current list of those.


% may be in an other note?
%
% UPDATE - licenses : type, uri as Datacite
% ADD - copyrights
% ADD - akcnowledgement
% ADD - curation text.. added values, column selection...., extract....
% list of Data Origin metadata  in registry

\section{Acknowledgment template}
\label{sect:template}

Example of acknowledgment that can be retrieved from the metadata:\todo{I don't think this section helps a lot; I'd drop it}

\textbf{Template}:\\
We extract data published in <related\_resource> (<creator>, <resource\_date>),
via <publisher> services (ivoa resource=<ivoid>, <publication\_date>)
using <service\_protocol> (version <version>, executed at <request\_date>\\


\textbf{Example}:\\
We extract data published in bibcode:2021AJ....161...36B (Bryson S., 2021),
via CDS services (ivoa resource=ivo://cds.vizier/j/aj/161/36, 2021-03-16)
using Simple Cone Search 1.03 (version 7.294, executed at 2022-10-30)



\appendix
\section{Appendix, Cone search serialization}\label{appendixA}
Simple Conesearch with its VOTable serialization. Data Origin are specified using  INFO.
\begin{verbatim}
<VOTABLE version="1.1" xmlns:xsi="http://www.w3.org/2001/XMLSchema-instance"
    xmlns="http://www.ivoa.net/xml/VOTable/v1.1" xsi:schemaLocation=...>

  <INFO name="protocol" value="Simple Cone Search 1.03"/><!-- URL or text ? -->
  <INFO name="request_date" value="2022-10-30T12:08:00"/>
  <INFO name="request"
   value="https://vizier.cds.unistra.fr/viz-bin/conesearch/J/AJ/161/36/table8?RA=285.4%26DEC=39.3%26SR=1"/>
  <INFO name="contact" value="cds-question@unistra.fr"/>
  <INFO name="version" value="7.294"/>


  <RESOURCE ID="yCat_51610036" name="J/AJ/161/36">
    <DESCRIPTION>117 exoplanets in habitable zone with Kepler DR25</DESCRIPTION>

    <INFO name="ivoid" value="ivo://cds.vizier/j/aj/161/36"/>
    <INFO name="publisher" value="CDS"/>
    <INFO name="landing_page"
          value="https://cdsarc.cds.unistra.fr/viz-bin/cat/J/AJ/161/36"/>

    <!-- Extra information from Data Origin (basic info) -->
    <INFO name="publication_id" value="doi:10.26093/cds/vizier.51610036"/>
    <INFO name="curation_level" value="Research"/>
    <INFO name="resource_version" value="2022-10-07"/>
    <INFO name="rights" value="https://cds.unistra.fr/vizier-org/licences_vizier.html"/>
    <INFO name="creator" value="Bryson S."/ ><!-- ORCID ? -->
    <INFO name="related_resource" value="2021AJ....161...36B"/>
    <INFO name="editor" value="Astronomical Journal"/>
    <INFO name="publication_date" value="2021-03-16"/>
    <INFO name="resource_date" value="2021"/>
    ....
    <TABLE>  .... </TABLE>
  </RESOURCE>
</VOTABLE>
\end{verbatim}

\section{Appendix, Changes from Previous Versions}

No previous versions yet.
% these would be subsections "Changes from v. WD-..."
% Use itemize environments.


\bibliography{ivoatex/ivoabib,ivoatex/docrepo}


\end{document}
