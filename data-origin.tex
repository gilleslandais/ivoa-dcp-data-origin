\documentclass[11pt,a4paper]{ivoa}
\input tthdefs

\title{Data Origin in the VO}

% see ivoatexDoc for what group names to use here
\ivoagroup{DCP}

%\author[????URL????]{G.Landais}
\author{G.Landais}
\author{G.Muench}
\author{looking for contributors}
%\author{????Fred Offline????}

\editor{G.Landais}

% \previousversion[????URL????]{????Concise Document Label????}
\previousversion{This is the first public release}
       

\begin{document}
\begin{abstract}
The goal of the document is to make the Data Origin more visible in the query results executed in the Virtual Observatory. 
The document lists meta-data required to provide sufficient traceability to end-users in order to improve the understanding 
of the resultsets and enabling its reuse and its citation.

\textbf{NOTE} in work -  template for a possible IVOA note.

\end{abstract}


\section*{Acknowledgments}


\section*{Conformance-related definitions}


\section{Introduction}

Data origin is required for end users to understand data, for citation and for reusability. The  provenance is cited as a mandatory criterion in the EOSC or in RDA FAIR definition. 

The virtual observatory provides an advanced framework to search and consume data provided by Data Centers or Space Agencies who apply curation in different level.  In this context, Data Origin in output includes meta-data from the data producer (author, space agency) and the Data Center which hosts the resource. 
Depending of the implementation, the users can find the origin information in the Data center web pages (landing pages) or in the Registries of the Virtual Observatory. For citation, ADS (Astrophysics Data System, Nasa) provides citation capabilities with bibtex output. There are no VO standards to get the information easily yet. For instance, the origin meta-data are not included neither in output format, nor in protocols used to access the data.
A list of basics meta-data added in strategic location (as result output or resource listing) would give easier the authors search who is looking for how to cite VO resources. Tracing data origin, from the producer to the final query enables also to report to end users the different agents implied in the data preservation (authors, data center, space agencies, journal)- especially when data can be subject to a curation  which depends of the different agents.
We propose to list the meta-data which responds to the need of Provenance and methods available today for their implementations. 

\subsection{Role within the VO Architecture}

%\begin{figure}
%\centering

% As of ivoatex 1.2, the architecture diagram is generated by ivoatex in
% SVG; copy ivoatex/archdiag-full.xml to role_diagram.xml and throw out
% all lines not relevant to your standard.
% Notes don't generally need this.  If you don't copy role_diagram.xml,
% you must remove role_diagram.pdf from SOURCES in the Makefile.

%\includegraphics[width=0.9\textwidth]{role_diagram.pdf}
%\caption{Architecture diagram for this document}
%\label{fig:archdiag}
%\end{figure}

%Fig.~\ref{fig:archdiag} shows the role this document plays within the
%IVOA architecture \citep{2010ivoa.rept.1123A}.

%???? and so on, LaTeX as you know and love it. ????



\section{Use cases}

\begin{itemize}
	\item (Data Origin information) A researcher has data in a VOTable that shows an odd feature. They would now like to talk to the creator of the data to help figure out whether that feature is physics or an artefact. 
	
	Requirement: contact information to producers present; but then let's not make that a MUST: This can be GDPR-relevant data, and it must be possible to leave it out if it is
	
	The researcher completes his understanding with Data Origin information easily accessible from the VOTable, and this, regardless of the service which generated the result. For instance, a URL that links an article. 
	
	The information could contain the Author, the year of publication, related resources like an article or the original data URL.
	
	When data provided by the service is derived from external resources, or if the data were performed with an additional curation, the nature and the links to the external resources are available.
	
	For instance, a table published in a journal or by a Space Agency is also hosted in a Data Center like CDS, GAVO, etc. The data curation depends of the Data Center which can add associated data, enrich meta-data (eg: add filter for magnitude) or make a sub-selection of columns. [an advanced serialisation could be based on DOI vocabulary "isVariantFormiOf", "IsDerivedFrom", ...]
	
	\item (Reproducibility) A researcher revisits work they did six months earlier in an ad-hoc fashion and would now like to reproduce it in a more structured fashion. Do do that, they need to know, say, which queries against which services, or perhaps which programs, produced the files. 
	
	Requirement: have the request parameters and a service identification (access url? ivoid?) in the data origin.
	
	\item (Citation) While preparing a publication, a researcher would like to properly cite the software and data that went into their results. They now run a program to extract that information from the digital artefacts going into the publication -- perhaps even in separate parts of citations and acknowledgements. 
	
	Requirement: The data origin must indicate requests for citation and/or acknowledgement in a machine-readable way, preferably in a way that machines can generate BibTeX for whatever they specify
	
	The information allows the researcher to fill the template citation asked by journals.
	
	Example (American Astronomical Society template):
	
	"we searched optical astrometric data of these sources from the Gaia (Gaia Collaboration et al. 2016) Early Data Release 3 (Gaia Collaboration et al. 2021) via the Gaia archive (Gaia Collaboration 2020)."*
	
	\item (workflow) Give me a bibliography of everything I've used in the workflow"
	The VOTable resulting of a session contains homogenized metadata that can be merged and compared.
	
\end{itemize}

The basics meta-data should contain the data origin (space agency or authors, article references), the data center providing the resource, the date of publication ...

\section{State of the art}

VOTable as ell as IVOA protocols don't provide any Data Origin information. For instance the TAP protocols provides tables and columns description in order to query tables. The TAP metadata contains information like unit, type and text information. Authors, publication date or identifiers are not included in the TAP description.

Hips is a more recent protocol which includes for each Dataset (HiPS) a list of standardized metadata. HiPS metadata include authors, publication year, Data center identifier or licenses.


\subsection{Data Origin in IVOA registry}
IVOA registries contains in its metadata the Data Origin like authors, publication date, references and alternate identifiers like DOI (eg: $<altIdentifier>doi:10.26093/cds/vizier.1355</altIdentifier>$).

The VO registries are available to index any services and resources hosted by data-producers, data-centers. 
The VO registries provide an open framework without any moderators. The data are regularly harvested by registries-of-registries which check the services/resources availability. However, the IVOA don't guaranty the resources sustainability.

The IVOA provides a unique identifier to any resources in the registry. The ivoid has not been considered to be citable in articles yet, because it is a technical identifier and because resources sustainability is not required in the IVOA publication process.

However, the registry metadata schema is based on Scholix and Datacite schemas. It opens citation capabilities with combining the metadata required to provide bibtex.

\subsection{Data Origin serialisation}
Today Provenance and DatasetDM are models that can be used to serialize Data Origin.

The Provenance Data Model (ProvDM) is based on Entities, Agents and Activities which are defined in the W3C Provenance. The model is dedicated to serialize workflow,
it is a recursive model that can be serialized in JSON or RDF. 

The ProvDM serialization needs mivot to be serialized in VOTable. But the recursive models and the metadata serialisation which is subject to the producer implementation, are obstacles for its reusability.


The "Last-Step-Provenance" is a Provenance extension (not a standard) which gather a list of metadata which matches with Data Origin. Its output is not recursive and could be easily serialized in a table.


DatasetDM provides a table description which includes Data Origin information like authors, date of publication, links to bibliography, etc.
DatasetDM is not a standard yet.

\subsection{DALI}
DALI is a protocol basis , used for instance in TAP. It includes services to  check the availability and capabilities of a service. It includes also information in the VOTable output format as the request STATUS or the request QUERY.


\section{Expected Data Origin}

List of expected metadata:

\begin{tabular}{|l|p{0.6\textwidth}|} \hline
Author & name or ORCID of the Data producer (human)\\ \hline
Organization&  name, ROR or URL of the Data producer (organisation)\\ \hline
Editor & name or URL  of the editor (when data are attached to a publication)\\ \hline
Journal & name or URL of the reference journal (when data are attached to a publication)\\ \hline
Datacenter & namen, ROR or URL of the data center who hosts the data \\ \hline
Contact & Data center email \\ \hline

Resource Identifier & ivoid of resource(s) hosted by the service which provides the result \\ \hline
Resource citation & DOI, bibcode of resource(s) hosted by the Data Center which returns the result \\ \hline
Original resource identifier & remote/original resource which was used to build the result \\ \hline
original resource citation & remote/original resource which was used to build the result \\ \hline
Publication date & publication date in the Data Center \\ \hline
Original Publication date &  publication date of the original resource \\ \hline
Curation level & curation level in the Data Center \\ \hline
Operation & Operation as cutout, add-values executed on Data-center on the original data \\ \hline
License & (original) licenses - machine-readable URI is preferred\\ \hline
Data version & version or release\\ \hline
Access protocol &  eg.TAP query, SCS, ... \\ \hline
Query & eg: ADQL \\ \hline
Version & version or date of Data-center software\\ \hline
.... & .... \\ \hline
\end{tabular}


\subsection{Condition for citation}
\textbf{I don't know if it is the place to talk about that??}


Data Citation requires a sustainable URL which is not guarantied in IVOA resources. 
Unlike ivoid, the DOI guaranties a sustainable URL and should be used for citation. 

Datacite provides a bibtex capability:
eg:


It could be possible to exploit ivoa registry metadata to provide a citation service for resources having a DOI.
In any cases, ADS (Nasa) citation is preferable for its clean curation and must be privileged each time it is possible.
However, ADS don't provide a citation for any datasets, even those having a DOI.





\section{Implementation tracks}

\subsection{VOTable proposal}

\subsection{Registry proposal}
\textbf{I don't know if it is the place to talk about that??}

% may be in an other note? 
%
% UPDATE - licenses : type, uri as Datacite
% ADD - copyrights
% ADD - akcnowledgement
% ADD - curation text.. added values, column selection...., extract....
% list of Data Origin metadata  in registry

\appendix
\section{Changes from Previous Versions}

No previous versions yet.  
% these would be subsections "Changes from v. WD-..."
% Use itemize environments.


% NOTE: IVOA recommendations must be cited from docrepo rather than ivoabib
% (REC entries there are for legacy documents only)



\bibliography{ivoatex/ivoabib,ivoatex/docrepo}


\end{document}
