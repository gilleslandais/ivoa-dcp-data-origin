\documentclass[11pt,a4paper]{ivoa}
\input tthdefs

\title{Data Origin in the VO}

% see ivoatexDoc for what group names to use here
\ivoagroup{DCP}

%\author[????URL????]{G.Landais}
\author{G.Landais}
\author{G.Muench}
\author{looking for contributors}
%\author{????Fred Offline????}

\editor{G.Landais}

% \previousversion[????URL????]{????Concise Document Label????}
\previousversion{This is the first public release}
       

\begin{document}
\begin{abstract}
The goal of the document is to make the Data Origin more visible in the query results executed in the Virtual Observatory. 
The document lists meta-data required to provide sufficient traceability to end-users in order to improve the understanding 
of the resultsets and enabling its reuse and its citation.

\textbf{NOTE} in work -  template for a possible IVOA note.

\end{abstract}


\section*{Acknowledgments}


\section*{Conformance-related definitions}


\section{Introduction}

Data origin is required for end users to understand data, for citation and for reusability. The  provenance is cited as a mandatory criterion in the EOSC or in RDA FAIR definition. 

The virtual observatory provides an advanced framework to search and consume data provided by Data Centers or Space Agencies who apply curation in different level.  In this context, Data Origin in output includes meta-data from the data producer (author, space agency) and the Data Center which hosts the resource. 
Depending of the implementation, the users can find the origin information in the Data center web pages (landing pages) or in the Registries of the Virtual Observatory. For citation, ADS (Astrophysics Data System, Nasa) provides citation capabilities with bibtex output. There are no VO standards to get the information easily yet. For instance, the origin meta-data are not included neither in output format, nor in protocols used to access the data.
A list of basics meta-data added in strategic location (as result output or resource listing) would give easier the authors search who is looking for how to cite VO resources. Tracing data origin, from the producer to the final query enables also to report to end users the different agents implied in the data preservation (authors, data center, space agencies, journal)- especially when data can be subject to a curation  which depends of the different agents.
We propose to list the meta-data which responds to the need of Provenance and methods available today for their implementations. 

\subsection{Role within the VO Architecture}

%\begin{figure}
%\centering

% As of ivoatex 1.2, the architecture diagram is generated by ivoatex in
% SVG; copy ivoatex/archdiag-full.xml to role_diagram.xml and throw out
% all lines not relevant to your standard.
% Notes don't generally need this.  If you don't copy role_diagram.xml,
% you must remove role_diagram.pdf from SOURCES in the Makefile.

%\includegraphics[width=0.9\textwidth]{role_diagram.pdf}
%\caption{Architecture diagram for this document}
%\label{fig:archdiag}
%\end{figure}

%Fig.~\ref{fig:archdiag} shows the role this document plays within the
%IVOA architecture \citep{2010ivoa.rept.1123A}.

%???? and so on, LaTeX as you know and love it. ????



\section{Use cases}

\begin{itemize}
	\item (Data Origin information) A researcher has data in a VOTable that shows an odd feature. They would now like to talk to the creator of the data to help figure out whether that feature is physics or an artefact. 
	
	Requirement: contact information to producers present; but then let's not make that a MUST: This can be GDPR-relevant data, and it must be possible to leave it out if it is
	
	The researcher completes his understanding with Data Origin information easily accesible from the VOtable, and this, regardless of the service which generated the result. For instance, a URL that links an article. 
	
	The information could contain the Author, the year of publication, related resources like an article or the original data URL.
	
	When data provided by the service is derived from external resources, or if the data were performed with an additional curation, the nature and the links to the external resources are available.
	
	For instance, a table published in a journal or by a Space Agency is also hosted in a Data Center like CDS, GAVO, etc. The data curation depends of the Data Center which can add associated data, enrich meta-data (eg: add filter for magnitude) or make a sub-selection of columns. [an advanced serialisation could be based on DOI vocabulary "isVariantFormiOf", "IsDerivedFrom", ...]
	
	\item (Reproducibility) A researcher revisits work they did six months earlier in an ad-hoc fashion and would now like to reproduce it in a more structured fashion. Do do that, they need to know, say, which queries against which services, or perhaps which programs, produced the files. 
	
	Requirement: have the request parameters and a service identification (access url? ivoid?) in the data origin.
	
	\item (Citation) While preparing a publication, a researcher would like to properly cite the software and data that went into their results. They now run a program to extract that information from the digital artefacts going into the publication -- perhaps even in separate parts of citations and acknowledgements. 
	
	Requirement: The data origin must indicate requests for citation and/or acknowledgement in a machine-readable way, preferably in a way that machines can generate BibTeX for whatever they specify
	
	The information allows the researcher to fill the template citation asked by journals.
	
	Example (American Astronomical Society template):
	
	"we searched optical astrometric data of these sources from the Gaia (Gaia Collaboration et al. 2016) Early Data Release 3 (Gaia Collaboration et al. 2021) via the Gaia archive (Gaia Collaboration 2020)."*
	
	\item (workflow) Give me a bibliography of everything I've used in the workflow"
	The VOTable resulting of a session contains homogenized metadata that can be merged and compared.
	
\end{itemize}

The basics meta-data should contain the data origin (space agency or authors, article references), the data center providing the resource, the date of publication ...

\section{State of the art}

VOTable as the most of IVOA protocols don't provide any Data origin information. For instance the TAP protocols provides tables and columns description in order togenerate and query tables. The TAP metadata contains information like unit, type and text information. Authors, ublication date or identifiers are not included in the TAP description.

The Hips server is a more recent protocol which includes for each dataset (a HiPS) a list of standardized metadata. Amon the metadata figures authors, publication yearsn, Data center idrntifier,or licences.


\subsection{Data-origin in IVOA registry}
The registry has Data origin like authorts, publication date, link to other resources and alternate identifier. Among the identifier, the DOI is accepted as alternate-identifier with a "doi" prefix (eg: $<altIdentifier>doi:10.26093/cds/vizier.1355</altIdentifier>$).

The VO registries offer services to any data-center, data-producers or service implementer to index a resource to be accessible in the Virtual Observatory framework. 
The IVOA registry is an open service without any moderators. The data are regularly harvested by registry-of registry which check the service/resource availability. However , the ivoa don't guaranty te resources sustainability.

The IVOID used to identify any resource in the registry is unique but it is not a identifier on sustatinable resource. 
A reason why the ivoid has not been considered yet to be citable in a scientific article. 

However, the registry metadata schema is based on schollix and datacite schemas. Then a curation registration could enable to provide the bibtex information to cite the data.

\subsection{Data Origin serialisation}
Today Proveance or DatasetDM provide models that can be used to serialize DataOrigin.

The Prorvenance DataModel (ProvDM) is based on Entities, Agents and Activities ofr the W3C Provenance description. The models has been done to serialize workflow.
It is a complete and recursiv model with json, rdf serialisation. Its implementation is open and depends of the data-producers implementation.

The ProvDM serialisation needs mivot to be serialised in VOTable. But its implementation too free and recusriv is a obstacle for its reusability.

The "Last-Step-Provenance" is an Provenance addon (not a standard) which lists a list of metadata which matches with Data-origin. Its output is not recursiv and could be easily seraialisezd in a table.


DatasetDM provides a table description which include DataOrigin information like authos, date odf publication, link to bibliography, etc.
DatasetDM is currently not a standard.


\section{Expected Data Origin}

\section{Implementation tracks}

\begin{itemize}
\item mapping metadata with Provenance, Datacite, ivo-registry ?...
\item Serialization using existing standard ? (ProvDM, mivot, VOTable, registry?)
\end{itemize}

\appendix
\section{Changes from Previous Versions}

No previous versions yet.  
% these would be subsections "Changes from v. WD-..."
% Use itemize environments.


% NOTE: IVOA recommendations must be cited from docrepo rather than ivoabib
% (REC entries there are for legacy documents only)



\bibliography{ivoatex/ivoabib,ivoatex/docrepo}


\end{document}
