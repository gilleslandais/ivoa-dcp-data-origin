\documentclass[11pt,a4paper]{ivoa}
\input tthdefs
\usepackage{todonotes}
\marginparwidth=4cm

\title{Data Origin in the VO}

% see ivoatexDoc for what group names to use here
\ivoagroup{DCP}

%\author[????URL????]{G.Landais}
\author{G.Landais}
\author{G.Muench}
\author{M.Demleitner}
\author{R.Savalle}
%\author{looking for contributors}
%\author{????Fred Offline????}

\editor{G.Landais}

%\previousversion{NOTE-20230522}
\previousversion{}
%\previousversion{This is the first public release}


\begin{document}
\begin{abstract}
Data Origin in the VO specifies a set of metadata items that define basic
provenance information, as well as their representation in documents produced
by Virtual Observatory (VO) services.  This will improve traceability for VO
users, help them understan result sets and facilitate data reuse and citation.

\end{abstract}


\section*{Acknowledgments}
Alberto Accomazzi (ADS), Anne Catherine Raugh (University of Maryland), Rafaele d'Abrusco (CfA), Mihaela Buga (CDS)

\section*{Conformance-related definitions}


\section{Introduction}

Information on the origin of a piece of data is important for end users to understand data, for meaningful data citation and to improved their reusability.  It is a part of provenance, which in turn is as a mandatory criterion in the RDA FAIR definition\todo{Please provide a link in a footnote}.

The Virtual Observatory (VO) provides an advanced framework to search for, query, and consume astronomical data.  The specification of Data Origin proposed here for VOTable output include both metadata originating at the data producer (e.g, author, space agency, observatory) and at the data centre (publisher) hosting the resource.

At this point, depending of the implementation, users can find the information conveyed in Data Origin in the data centre web pages (landing pages) or in the VO Registry.  For citation, the ADS (NASA Astrophysics Data System) offers comprehensive bibliographic capabilities, including the production of BibTeX records for publications known to ADS.  However, there are no VO standards to communicate this type of information \emph{in instance documents} yet.

A list of basic metadata added reliably findable in a convenient location (i.e.,
the VOTable produced by a query) will help users to properly cite or
acknowledge the data resources contributing to new or derived works.
Tracing Data Origin, from the producer to the query to the production of the respons, also allows an end users to determine the different agents implied in data preservation (authors, data centre, space agencies, journal), which is particularly helpful when debugging.  A typical scenario here is when mirrored data is be subject to potentially differing curation actions in the different publication processes.

The list of metadata items proposed here is designed to meet the needs of basic provenance
tracking when using current VO protocols.

The remainder of this paper first presents the use cases guiding this
effort, then briefly lists the sort of metadata Data Origin concerns.
The core of the specification is in
section~\ref{sec:data-origin-in-votable}, which
gives the VOTable serialisation.\todo{As I said, I think I'd rather skip
sects. 6 and 7 or make them appendices, but if you really want to keep
them, say something about them here}

\section{Use cases}

\subsection{Data Origin information} 

Scenario: Researchers have data in a VOTable that shows an odd feature. They would now like investigate whether that feature is physical or an artefact.

Derived requirements:

\begin{itemize}
\item Contact information for producers must be present.
	
\item Researchers would like to identify clearly defined roles in well-known places regardless of the service which generated the result.
		
\item When data provided by the service is derived from external resources, these external resources are clearly identified.  In that case, additional curation applied by the publisher can be detected.
\end{itemize}

For instance, a table published in a journal or by a space agency is also hosted in multiple data centre. The details of the table schema may depend on the data centre, which can add associated data, enrich metadata, or make a sub-selection of columns.


\subsection{Reproducibility} 

A researcher revisits work they did six months earlier in an ad-hoc fashion and would now like to reproduce it in a more structured way. To do that, they need to know, say, which queries against which services, or perhaps which programs, produced the files.
	
Derived requirement: The request parameters and a service identification
(perhaps an ivoid in a narrower VO context, an access URL beyond that) must be available.
	
\subsection{Citation} 
\label{sec:req-citation}

While preparing a publication, a researcher would like to properly cite the software and data that went into their results. They now run a program to extract that information from the digital artefacts going into the publication -- perhaps even in separate parts of citations and acknowledgments.

The type of information that would go into such a
citation can be assessed from this acknowledgement following a
convention of the American Astronomical Society:

\begin{quotation}
We searched optical astrometric data of these sources from the Gaia (Gaia Collaboration et al. 2016) Early Data Release 3 (Gaia Collaboration et al. 2021) via the CDS archive.
\end{quotation}

Derived requirement: The Data Origin must indicate requests for citation
and/or acknowledgment in a machine-readable way, preferably in a way
that machines can generate BibTeX for whatever they specify.
	
	
\subsection{Workflow bibliography} 

A resarcher has used a workflow engine to solve a complex science
problem.  They now want to create a bibliography of everything that was
used to obtain the end result of the computation.

Derived requirement: Essentially as for use case~\ref{sec:req-citation},
except that the metadata extracted needs a higher level of homogeneity
and that relationships declared between different parts of Data Origin
might be useful.


\section{State of the art}

Neither VOTable \citep{2019ivoa.spec.1021O} nor other IVOA protocols provide any Data Origin information. For instance, the TAP protocol \citep{2019ivoa.spec.0927D} provides tables and columns description in order to query tables. The TAP metadata contains information like unit, type and text information. Authors, publication date or identifiers are not included in the TAP description.

HiPS \citep{2017ivoa.spec.0519F} is a more recent protocol which includes for each Dataset (HiPS) a list of standardized metadata. HiPS metadata include authors, publication year, data centre identifier or licenses.


\subsection{Data Origin in IVOA registry}
The IVOA Registry contains metadata relevant for Data Origin, for instance, authors, publication date, references and alternate identifiers like DOI for each of its resources \citep{2018ivoa.spec.0625P}.\\
e.g.: \xmlel{<altIdentifier>}doi:10.26093/cds/vizier.1355\xmlel{</altIdentifier>} \\
It makes this information available through several interfaces, partly
hosted by the data centres themselves, partly provided by a central
infrastructure.
The VO Registry is an open framework without any moderators.
The IVOA hence does not guaranty the resources' sustainability.

The IVOA Registry provides a unique identifier (the ivoid; see \citet{2016ivoa.spec.0523D}) for each resource.  This ivoid has not been considered to be citable in articles, because it is a technical identifier with no provisions for persistence.

Both the Registry's metadata schema and the DataCite
\citep{std:DataCite40} metadata schema have been
derived from Dublin Core \citep{std:DUBLINCORE}.  While the extensions differ in detail, it is not
hard to write mapping between the two metadata schemas.  This can enable
sustainable citation.

\subsection{Data Origin serialisation}
The Provenance \citep{2020ivoa.spec.0411S} and Dataset Data Models can be used to serialize Data Origin.

The Provenance Data Model (ProvDM) is based on Entities, Agents and Activities which are defined in the W3C Provenance. The model is dedicated to serialize workflow,
it is a recursive model that can be serialized in JSON or RDF.

The ProvDM serialization needs mivot to be serialized in VOTable. But the recursive model and the metadata serialization which is subject to the producer implementation, are obstacles for its reusability.


The "Last-Step-Provenance" is a Provenance extension (not a standard yet) which gather a list of metadata which matches with Data Origin. Its output is not recursive and could be easily serialized in a table.


DatasetDM provides a table description which includes Data Origin information like authors, date of publication, links to bibliography, etc.
DatasetDM is not a standard yet.

\subsection{DALI}
DALI \citep{2017ivoa.spec.0517D} is a basis for several VO protocols, used for instance in TAP. It includes services to  check the availability and capabilities of a service. It includes also information in VOTable as the request STATUS or the REQUEST query.

It already defines bespoke names for \xmlel{INFO} elements used to convey additional metadata, in particular \emph{citation}.  In a sense, this specification is an extension of the mechanism defined in DALI.

\section{Expected Data Origin}
% added 23-nov-2023
This document lists metadata expected to be accessible for users that consume data in the VO.

It includes reproducibility metadata (see \ref{tab:query-names}) that reflects the context in which a query was executed. The information includes parameters allowing to execute the query again as well as other parameters, like version or execution date applied to resources that may evolved.


The information is completed by provenance metadata (see \ref{tab:origin-names}) like authors, licence, references, identifiers (eg: ivoid, bibcode, doi).

All of these provenance metadata are generally provided through the registry. Access to the information is simplified by adding the ivoid for each resources, or by encapsulating the information directly in the VO output (see \ref{sec:data-origin-in-votable}).

% end

% remove 23-nov-2023

%%List of expected metadata:
%%
%%\begin{itemize}
%%\item Author -- name or ORCID of the Data producer (human)
%%\item Organization--  name, ROR or URL of the Data producer (organization)
%%\item Editor -- name or URL  of the editor (when data are attached to a publication)
%%\item Journal -- name or URL of a journal (or similar aggregation of publications) holding a publication on data that was used to build the current resource
%%\item Data centre -- name, ROR or URL of the data centre who hosts the data
%%\item Contact -- data centre email
%%\item Resource Identifier -- ivoid of resource(s) hosted by the service which provides the result
%%\item Resource citation -- DOI, bibcode of resource(s) hosted by the data centre which returns the result
%%\item Original resource identifier -- remote/original resource which was used to build the result
%%\item Publication date -- publication date in the data centre
%%\item Original Publication date --  publication date of the original resource
%%\item Curation level -- curation level in the data centre
%%\item Operation -- Operation as cutout, add-values executed on Data-centre on the original data
%%\item License -- (original) licenses - an SPDX\footnote{\url{https://spdx.dev/}}
%%URI is preferred
%%\item Data version -- version or release
%%\item Access protocol --  e.g., TAP, SCS, \dots
%%\item Query -- In protocols like TAP, the query passed in by the user as
%%as single string
%%\item Version -- version or date of Data-centre software
%%%\item ... (\textbf{to complete?})
%%\end{itemize}

% end

\subsection{Condition for citation}
%\todo{I don't know if it is the place to talk about that??}

The DOI is the privileged persistent identifier to cite resources.\\

%Data Citation requires a sustainable URL which is not guarantied in IVOA resources.
%Unlike ivoid, the DOI guaranties a sustainable URL and should be used for citation. \\
Data citation requires a persistent identifier and a sustainable URL.
Both are guaranteed by DOI, but resource provided with an ivoid (the IVOA identifier)
is not guaranteed to be sustainable.\\

Bibtex requires curation that needs metadata like identifier, authors, title, publisher and date of publication.
ADS (NASA) provides a citation capability for its indexed resources. This curation quality has to be privileged or may be took as example for any data providers and users.

For instance, DOI providers like Datacite, provides a bibtex capability. The bibtex quality depends of the DOI metadata filled by the Data producers and publishers.\\

The IVOA registry which contains metadata for any resources could be used to get the expected quality for citation if the following conditions are met:
\begin{itemize}
\item the registry resource includes a persistent identifier (DOI)
\item the registry resource includes the metadata which meets the bibtex requirements
\end{itemize}

% move metadata list : 23-nov-2023


\section{Data Origin in VOTable}
\label{sec:data-origin-in-votable}
The metadata listed bellow combines terms from DALI (REC-DALI-1.1), Dublin Core (DC: Dublin core, RFC 2413) and extensions in order to reproduce and to provide Data Origin information to end users.

\subsection{Query information}
Table~\ref{tab:query-names} lists the metadata items defined here to
convey query-related information in Data Origin.

These pieces of information enable linking Registry records to the
current result and, to some extent, reproducing the query executed. For
queries on evolving datasets, the version or the date must complete the
information.

\begin{table}
\begin{tabular}{|l|l|l|l|}  \hline
\textbf{metadata} & \textbf{Description} & \textbf{Level} & \textbf{Dublin Core}\\ \hline
ivoid             & ivoid identifier to link registry & R &  \\ \hline
publisher         & data centre that provides the VOTable & R & publisher\\ \hline
%rename 23-nov-2023 version           & Software version (*) & & \\ \hline
server\_software  & Software version (*) & & \\ \hline
service\_protocol & Protcol access with version & R& \\ \hline
request           & Request url (**)&  R& \\ \hline
query             & Query readable by human (e.g.: ADQL) &  & \\ \hline
% removed in 23-nov-2023
%request\_post     & (POST Request) POST arguments &  & \\ \hline
% end
request\_date     & Query execution date & R&\\ \hline	
contact           & email or URL to contact publisher & & \\ \hline	
landing\_page     & Dataset landing page & & \\ \hline
\multicolumn{4}{l}{\footnotesize(*) Free text, standardized semantic versioning encouraged, see \url{https://semver.org/}} \\
\multicolumn{4}{p{\textwidth}}{\footnotesize(**) For "Simple" protocols using POST, put in the application/x-www-form-urlencoded form of the query parameters. multipart/form-data is not explored in this current document.}
\end{tabular}
\caption{\xmlel{INFO} names available for specifying the query that
generated a VOTable}
\label{tab:query-names}
\end{table}




\subsection{Dataset Origin}
Dataset origin complements the query-related information to improve the
understandability of the underlying data. This information is intended
for end users.  If the resource is also described in the Registry, care
must be taken that in-response metadata remains in sync with metadata
available there.


Table~\ref{tab:origin-names} lists the origin-related metadata items
defined here.

\begin{table}
\begin{tabular}{|l|p{5cm}|l|l|}  \hline
\textbf{metadata} & \textbf{Description} & \textbf{Level} & \textbf{Dublin Core}\\ \hline
% removed 23-nov-2023 publication\_id    & Dataset identifier that can be used for citation& M  & identifier\\ \hline
citation    & Dataset identifier that can be used for citation& R  & identifier\\ \hline
% removed in 23-nov-2023
%curation\_level    & Controled vocabulary
%                   (IVOA rdf, content\_level) &  &  \\ \hline
% end
% modifier 23-nov-2023 resource\_version  & Dataset version or last release & R & \\ \hline
resource\_version  & Dataset version  & R & \\ \hline
%rename 23-nov-2023 rights             & (*) Licence URI & R & rights\\ \hline
rights\_uri        & (*) Licence URI & R & rights\\ \hline
% removed 23-nov-2023 rights\_type       & (*) Licence type (eg: CC-by, CC-0, private, public) &  &  \\ \hline
%rename 23-nov-2023 copyrights         & Copyright text &  & \\ \hline
rights             & Licence or Copyright text &  & rights\\ \hline
creator            & The person or organization primarily responsible for creating the
                     intellectual content of the resource.  For example, authors in the
                     case of written documents, artists, photographers, or illustrators in
                     the case of visual resources. & R & creator\\ \hline
editor             & Editor name &  & \\ \hline
relation\_type     & An identifier of a second resource and its relationship to the
                     present resource.
                     Controlled vocabulary (**)& & relation\\ \hline
related\_resource  & Information about a second resource from which the present resource
                     is derived. The source is an identifier that can be prefixed with the identifier type: eg: bibcode:, doi:, ror: &  & source\\ \hline
% remove 23-nov-2023
%publication\_date  & Date of publication (format ISO 8601) &  R &  \\ \hline
%resource\_date     & Date of the original publication (format ISO 8601) & R & date\\ \hline
%
original\_date     & Date of the original resource from which the present resource is derived. (format ISO 8601) &    &  \\ \hline
publication\_date  & Date of first publication in the Data Center (***) (format ISO 8601) &  R &  \\ \hline
last\_update\_date & Last Data Center update (****) (format ISO 8601) & R & date\\ \hline
\multicolumn{4}{p{\textwidth}}{\footnotesize(*) Priviledge Machine-readable Licence.
See SPDX list \url{https://spdx.org/licenses/} or Creative Commons licenses \url{https://creativecommons.org}}\\
\multicolumn{4}{p{\textwidth}}{\footnotesize(**) \url{https://www.ivoa.net/rdf/voresource/relationship\_type/}}\\
\multicolumn{4}{p{\textwidth}}{\footnotesize(***) Equivalent to curation/date[@role='created'] in registry}\\
\multicolumn{4}{p{\textwidth}}{\footnotesize(****) Equivalent to curation/date[@role='updated'] in registry}
\end{tabular}
\caption{\xmlel{INFO} names available for specifying information
related to the origin of the data set(s) a VOTable was generated from}
\label{tab:origin-names}
\end{table}


\subsection{VOTable serialization}

In this document, we focused on a basic serialization that allows description implying individual tables.
This output is adapted for protocol like the Simple Conesearch.

The basic serialization uses INFO tags to populate Data Origin (see the example of a ConeSearch result in appendix  \ref{appendixA}).
INFO tags are allowed in VOTable under \xmlel{VOTABLE} or in \xmlel{RESOURCE} elements.
Thus, it becomes possible to annotate a collection of TABLE which are in different resources.

This specification at this point does not constrain the multiplicities of individual INFO items, and clients should not fail hard if any given INFO item occurs multiple times.

Complex query (for instance, a join in a TAP query) needs an advanced output serialization to gather metadata by resource.
Mechanisms to manage this requirement are being developed in IVOA (mivot).
This output is not treated in the current version of the document.

As a service to human readers, it is recommended to put descriptions, possibly derived from definitions provided in this document, into the bodies of the INFO elements.


\section{Data Origin in Registry}
The ivo-id, now available in VOTable, allows to query the resource metadata which are in the VO registry.\\


Expected metadata (VOResource) with their equivalent in Datacite schema (version 4.4) to provide Data Origin in the registry\\

\begin{tabular}{|p{3cm}|p{4cm}|p{1cm}|p{5cm}|} \hline
\textbf{VOResource} & \textbf{DataCite} & \textbf{Level} & \textbf{Explain} \\ \hline
identifier    &Identifier (1) &M & ivoid of resource(s) hosted by the service\\ \hline
title         &Title (3) &M  & resource title\\ \hline
shortName     &&& Resource short name\\ \hline
altIdentifier & AlternateIdentifier (11)& R &
              Alternate identifier accepts bibcode, DOI or URL. DOI should be privileged to facilitate citation and link with DataCite or Crossref..eg: DOI \\ \hline
\end{tabular}

\begin{tabular}{|p{3cm}|p{4cm}|p{1cm}|p{5cm}|} \hline
\multicolumn{4}{|l|}{\textbf{curation}} \\ \hline
publisher     & Publisher (4) & M &publisher (*)\\ \hline
creator       & Creator (2) & M & author(s) (*)\\ \hline
contributor   & Contributor & & contributor(s) (*)\\ \hline
date [Created]& Dates [created] (8)& M & creation date (in data centre)\\ \hline
date [Updated]& Dates [updated] (8)& M & last modification\\ \hline
  ?            & PublicationYear (5) & & publication year in data centre\\ \hline
version       & Version (15) & R &\\ \hline
contact       & &&\\ \hline
\multicolumn{4}{l}{\small \footnotesize(*) terms allowing name and Orcid (AltIdentifier in VOResurce)} \\
\end{tabular}

\begin{tabular}{|p{3cm}|p{4cm}|p{1cm}|p{5cm}|} \hline
\multicolumn{4}{|l|}{\textbf{content} } \\ \hline
source        & RelatedIdentifier (12) (*) & R & bibcode\\ \hline
referenceURL  & & R & landing page\\ \hline
type          & ResourceType (10)& & Resource type (catalog, etc)\\ \hline
description   & Description (17)& & abstract\\ \hline
contentLevel  & & &\\ \hline
relationShip  & RelatedIdentifiers (12) & R &link to remote resource (Recommended to link Original data centre) \\ \hline
relationshipType & relationType (12.b) & &\\ \hline
relatedResource  & RelatedIdentifier (12) & R &\\ \hline
\multicolumn{4}{l}{\small \footnotesize(*) DataCite sub-properties type=bibcode, relationType=IsSupplementTo} \\
\end{tabular}

\begin{tabular}{|p{3cm}|p{4cm}|p{1cm}|p{5cm}|} \hline
\multicolumn{4}{l}{\textbf{rights}} \\ \hline
rights   & Rights (16)& R& license
          The right element accepts free text. However, it is preferable to provide a machine-readable License (*)
         \\ \hline
URI     & rightsURI & R& license URL\\ \hline
         & rightsIdentifier & & standard license name .ex CC-by.
         Copyright is accepted by FAIR principle. But copyright is only a link to the data producer. It gives the contact point to any users who would like to use data. Copyright is more simple to implement for data-centre that provides a copy of original resource, but its use is not well integrated in an interoperable workflow.
         \\ \hline
\multicolumn{4}{p{\textwidth}}{\small \footnotesize(*) See SPDX list \url{https://spdx.org/licenses/} or Creative Commons licenses \url{https://creativecommons.org}}
\end{tabular}\\




\textbf{Examples}

Examples of rights serialization:

\begin{verbatim}
<rights rightsURI="https://spdx.org/licenses/ODbL-1.0.html">
  ODbL-1.0
</rights>
\end{verbatim}

\begin{verbatim}
<rights rightsURI="https://creativecommons.org/licenses/by/4.0/">
  Creative Commons Attribution 4.0
</rights>
\end{verbatim}


Example or relation ship :
Cite the original dataset using "source" (to link a bibliographic reference) or "relatedIdentifier" (to link a dataset)

e.g.:
\begin{verbatim}
<relationship>
  <relationshipType>Cites</relationshipType>
  <relatedResource>doi: 10.5270/esa-qa4lep3 : Gaia DR3 ESA</relatedResource>
</relationship>
\end{verbatim}

Example of Creator:
\begin{verbatim}
<creator>
  <name>Bryson S.</name>
  <altIdentifier>orcid:0000-0003-0081-1797<altIdentifier>
</creator>
\end{verbatim}

% may be in an other note?
%
% UPDATE - licenses : type, uri as Datacite
% ADD - copyrights
% ADD - akcnowledgement
% ADD - curation text.. added values, column selection...., extract....
% list of Data Origin metadata  in registry

\section{Citation Template}

Example of citation template that could be included in authors articles. The template exploits the metadata described in this document:\\

\textbf{Template}:\\
We extract data published in <related\_resource> (<creator>, <original\_date>),
via <publisher> services (ivoa resource=<ivoid>, <publication\_date>)
using <service\_protocol> (version <server\_software>, executed at <request\_date>\\


\textbf{Example}:\\
We extract data published in bibcode:2021AJ....161...36B (Bryson S., 2021),
via CDS services (ivoa resource=ivo://cds.vizier/j/aj/161/36, 2021-03-16)
using Simple Cone Search 1.03 (version 7.294, executed at 2022-10-30)



\appendix
\section{Appendix, Cone search serialization}\label{appendixA}
Simple Conesearch with its VOTable serialization. Data Origin are specified using  INFO.
\begin{verbatim}
<VOTABLE version="1.1" xmlns:xsi="http://www.w3.org/2001/XMLSchema-instance"
    xmlns="http://www.ivoa.net/xml/VOTable/v1.1" xsi:schemaLocation=...>

  <INFO name="protocol" value="Simple Cone Search 1.03"/><!-- URL or text ? -->
  <INFO name="request_date" value="2022-10-30T12:08:00">Query execution date</INFO>
  <INFO name="request"
value="https://vizier.cds.unistra.fr/viz-bin/conesearch/J/AJ/161/36/table8?RA=28.4%26DEC=39.3%26SR=1"/>
  <INFO name="contact" value="cds-question@unistra.fr">Publisher contact</INFO>
  <INFO name="server_software" value="7.294">Software version</INFO>


  <RESOURCE ID="yCat_51610036" name="J/AJ/161/36">
    <DESCRIPTION>117 exoplanets in habitable zone with Kepler DR25</DESCRIPTION>

    <INFO name="ivoid" value="ivo://cds.vizier/j/aj/161/36">
        ivoid identifier to link registry
    </INFO>
    <INFO name="publisher" value="CDS">data centre</INFO>
    <INFO name="landing_page"
          value="https://cdsarc.cds.unistra.fr/viz-bin/cat/J/AJ/161/36"/>

    <!-- Extra information from Data Origin (basic info) -->
    <INFO name="citation" value="doi:10.26093/cds/vizier.51610036">
        Identifier that can be used for citation
    </INFO>
    <INFO name="last_update_date" value="2022-10-07">Last Data Center update</INFO>
    <INFO name="rights"
          value="https://cds.unistra.fr/vizier-org/licences_vizier.html"/>
    <INFO name="creator" value="Bryson S.">Author</INFO><!-- ORCID ? -->
    <INFO name="related_resource" value="2021AJ....161...36B">
        Reference article
    </INFO>
    <INFO name="editor" value="Astronomical Journal">
        Journal of the reference article
    </INFO>
    <INFO name="publication_date" value="2021-03-16">
        Date of first publication in the Data Center
    </INFO>
    <INFO name="original_date" value="2021">Publication Date of the article</INFO>
    ....
    <TABLE>  .... </TABLE>
  </RESOURCE>
</VOTABLE>
\end{verbatim}

\section{Appendix, Changes from Previous Versions}

%No previous versions yet.
% these would be subsections "Changes from v. WD-..."
% Use itemize environments.
\subsection{Data Origin in the VO Version 1.0}
\begin{itemize}
\item Re-organize the section 4 and move items information into section 5.
\item Changed vocabulary: \\
  \textit{resource\_date} becomes \textit{last\_update\_date},
  \textit{rights} and \textit{copyrights} becomes \textit{rights\_uri} and \textit{rights} (VOResource),
  \textit{version} becomes \textit{server\_software} (https://ivoa.net/documents/Notes/softid/),
  \textit{publication\_id} becomes \textit{citation} (DALI)
\item Remove \textit{curation\_level}, \textit{request\_post}, \textit{rights\_type}
\item New item: \textit{original\_date}
\item Clarify date items and multiple INFO in VOTable.
\item Propose added human-readable text for VOTable serialization.
\end{itemize}

\bibliography{ivoatex/ivoabib,ivoatex/docrepo}


\end{document}
